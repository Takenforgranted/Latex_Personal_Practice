\documentclass[UTF8]{ctexart}
\usepackage{amsmath}

\begin{document}
\section{题目}
设 $f:\{1,2,\dots,n\} \rightarrow \{1,2,\dots,n\}$ 为双射,证明:

\[
\sum_{k=1}^{n} \frac{a_{f(k)}}{a_k} \geq n
\]

其中 $a_1, a_2, \dots, a_n$ 均为实数。

\section{解答}
解:根据康托罗维奇不等式,我们有:

\[
\left( \sum_{k=1}^{n} \frac{a_{f(k)}}{\sqrt{a_k}} \cdot \sqrt{a_k} \right)^2 \leq \left( \sum_{k=1}^{n} \left( \frac{a_{f(k)}}{\sqrt{a_k}} \right)^2 \right) \left( \sum_{k=1}^{n} a_k \right)
\]

化简上述不等式,得到:

\[
\left( \sum_{k=1}^{n} \frac{a_{f(k)}}{\sqrt{a_k}} \cdot \sqrt{a_k} \right)^2 \leq \left( \sum_{k=1}^{n} \frac{a_{f(k)}^2}{a_k} \right) \left( \sum_{k=1}^{n} a_k \right)
\]

由于$f$是双射,我们可以对右侧的两个求和符号进行重新排列。

\[
\left( \sum_{k=1}^{n} \frac{a_{f(k)}}{\sqrt{a_k}} \cdot \sqrt{a_k} \right)^2 \leq \left( \sum_{k=1}^{n} \frac{a_{k}}{a_k} \right) \left( \sum_{k=1}^{n} a_k \right)
\]

继续化简,我们得到:

\[
\left( \sum_{k=1}^{n} \frac{a_{f(k)}}{\sqrt{a_k}} \cdot \sqrt{a_k} \right)^2 \leq n \left( \sum_{k=1}^{n} a_k \right)
\]

因为$a_k$是实数,所以根据不等式平方的性质,我们可以得到:

\[
\sum_{k=1}^{n} \frac{a_{f(k)}}{\sqrt{a_k}} \cdot \sqrt{a_k} \geq \sqrt{n \left( \sum_{k=1}^{n} a_k \right)}
\]

最后,我们对不等式两边同时平方,得到:

\[
\left( \sum_{k=1}^{n} \frac{a_{f(k)}}{a_k} \right) \left( \sum_{k=1}^{n} a_k \right) \geq n \left( \sum_{k=1}^{n} a_k \right)
\]

这进一步简化为:

\[
\sum_{k=1}^{n} \frac{a_{f(k)}}{a_k} \geq n
\]

\section*{附录}

\subsection*{康托罗维奇不等式的证明}

康托罗维奇不等式是数学中一条重要的不等式,它可以用来证明许多其他的数学结果。下面我们给出康托罗维奇不等式的证明:

设有$n$个实数$a_1, a_2, \dots, a_n$以及$b_1, b_2, \dots, b_n$为$n$个实数的任意排列,即$(b_1, b_2, \dots, b_n)$是$(a_1, a_2, \dots, a_n)$的一个重排列。

我们考虑两个向量$ \mathbf{a} = \begin{pmatrix} a_1 \\ a_2 \\ \vdots \\ a_n \end{pmatrix} $ 和 $\mathbf{b} = \begin{pmatrix} b_1 \\ b_2 \\ \vdots \\ b_n \end{pmatrix} $。它们的内积定义为 $\mathbf{a} \cdot \mathbf{b} = \sum_{k=1}^{n} a_k b_k$。

根据柯西-施瓦茨不等式,我们有 $\left( \sum_{k=1}^{n} a_k b_k \right)^2 \leq \left( \sum_{k=1}^{n} a_k^2 \right) \left( \sum_{k=1}^{n} b_k^2 \right)$。

取到$b_k$为$a_1, a_2, \dots, a_n$的一个重排列,我们可以得到:

\[
\left( \sum_{k=1}^{n} a_k b_k \right)^2 \leq \left( \sum_{k=1}^{n} a_k^2 \right) \left( \sum_{k=1}^{n} b_k^2 \right)
\]

由于$b_1, b_2, \dots, b_n$是$a_1, a_2, \dots, a_n$的一个重排列,那么$b_1, b_2, \dots, b_n$也可以看作是$f(1), f(2), \dots, f(n)$的一个重排列,其中$f:\{1,2,\dots,n\} \rightarrow \{1,2,\dots,n\}$是一个双射。即存在一个双射$f$,使得$b_k=a_{f(k)}$。

将上述等式改写为:

\[
\left( \sum_{k=1}^{n} a_k a_{f(k)} \right)^2 \leq \left( \sum_{k=1}^{n} a_k^2 \right) \left( \sum_{k=1}^{n} a_{f(k)}^2 \right)
\]

进一步展开,我们得到:

\[
\left( \sum_{k=1}^{n} a_k a_{f(k)} \right)^2 \leq \left( \sum_{k=1}^{n} a_k^2 \right) \left( \sum_{k=1}^{n} a_{f(k)}^2 \right)
\]

因为$f$是双射,我们可以对右边的两个求和符号进行重新排列:

\[
\sum_{k=1}^{n} a_k a_{f(k)} \leq \sqrt{\left( \sum_{k=1}^{n} a_k^2 \right) \left( \sum_{k=1}^{n} a_{f(k)}^2 \right)}
\]

继续化简,我们得到:

\[
\frac{\sum_{k=1}^{n} a_{f(k)}}{\sqrt{\sum_{k=1}^{n} a_k^2}} \geq 1
\]

由于$a_1, a_2, \dots, a_n$是实数,所以我们可以将其写成绝对值的形式:

\[
\left| \sum_{k=1}^{n} \frac{a_{f(k)}}{\sqrt{a_k^2}} \right| \geq 1
\]

进一步化简,我们得到康托罗维奇不等式:

\[
\left| \sum_{k=1}^{n} \frac{a_{f(k)}}{a_k} \right| \geq 1
\]

其中$f:\{1,2,\dots,n\} \rightarrow \{1,2,\dots,n\}$是一个双射。

这就完成了康托罗维奇不等式的证明。
\end{document}
